
\section{Basic Exponential Family Theory}

The \emph{Laplace transform} of a positive measure $\lambda$ on $\real^J$
is \citep[Chapter~7]{barndorff} the function $c : \real^J \to (0, \infty]$
defined by
$$
   c(\boldvarphi) = \int e^{\inner{\boldx, \boldvarphi}} \lambda(d \boldx).
$$
(note that the value $+ \infty$ is allowed so the function is defined for
all $\boldvarphi$).  A log Laplace transform is both convex and lower
semicontinuous \citep[Theorem~7.1]{barndorff}.  This implies that
$$
   \Phi = \set{ \boldvarphi \in \real^J : c(\boldvarphi) < \infty }
$$
is a convex set.

The full standard exponential family generated by $\lambda$ is the family
$$
   \mathcal{P} = \set{ P_\boldvarphi :  \boldvarphi \in \Phi }
$$
where $P_\boldvarphi$ is the distribution having density with respect to
$\lambda$ defined by
$$
   f_\boldvarphi(x) = \frac{1}{c(\boldvarphi)} e^{\inner{\boldx, \boldvarphi}},
   \qquad \boldvarphi \in \Phi
$$
\citep[Chapter~8]{barndorff}.  The moment generating function
of $P_\boldvarphi$ is defined by
$$
   M_\boldvarphi(t)
   =
   \int e^{\inner{\boldx, t}} P_\boldvarphi(d \boldx)
   =
   \frac{c(\boldvarphi + t)}{c(\boldvarphi)}
$$
From this we see that a Laplace transform is just like a moment generating
function, except for a general positive measure $\lambda$ instead of for a
probability measure $P_\boldvarphi$.

The \emph{cumulant generating function} is the log of the moment generating
function.  Its derivatives evaluated at $t = 0$ are the \emph{cumulants}
of the distribution (the first two are mean and variance).  Since the
derivatives of $\log M_\varphi$ evaluated at $t = 0$ are the same as the
derivatives of $\psi = \log c$ evaluated at $\boldvarphi$, we call $\psi$
the \emph{cumulant function} of the family.

It is sometimes convenient to choose for the dominating measure of the family
($\lambda$ in our original notation) one of the distributions in the family,
a $P_{\boldvarphi^*}$ for some $\boldvarphi^* \in \Phi$.
We need to see what that does to our
original formulation with $\lambda$ as the dominating measure.

The density of $P_\boldvarphi$ with respect to $P_{\boldvarphi^*}$ is
just the ratio of their densities with respect to $\lambda$
\begin{equation} \label{eq:move}
   g_\boldvarphi(x)
   =
   \frac{f_\boldvarphi(x)}{f_{\boldvarphi^*}(x)}
   =
   \frac{c(\boldvarphi^*)}{c(\boldvarphi)}
   e^{\inner{x, \boldvarphi - \boldvarphi^*}}
\end{equation}
and we see that the Laplace transform for this ``new'' family is
$c(\boldvarphi) / c(\boldvarphi^*)$ and the cumulant function is
$\psi(\boldvarphi) - \psi(\boldvarphi^*)$.

\section{Exponential Family Theory Applied to Aster Models}

\subsection{Cumulant Function and Full Canonical Parameter Space}

This appendix investigates what happens when the canonical parameter spaces
of an aster model are not all of $\real^d$.  The full \emph{conditional}
canonical parameter space is a Cartesian product
$$
   \Theta = \prod_{j \in J} \Theta_j
$$
where $\Theta_j$ is the full canonical parameter space of the
conditional one-parameter exponential family model at the $j$-th node
(which has cumulant function $\psi_j$), the set
$$
   \Theta_j = \set{ \theta \in \real : \psi_j(\theta) < \infty }.
$$

Let $P_{\theta, j}$ be the conditional probability measure of the $j$-th
family.  So the joint distribution is
$$
   P_{\boldtheta}(d \boldx)
   =
   \prod_{j \in J} P_{\theta_j, j}(d x_j \mid x_{p(j)}).
$$

Fix $\boldtheta^* \in \Theta$ and let $\boldvarphi^*$ be the corresponding
unconditional canonical parameter found by applying the map \eqref{eq:new}
to $\boldtheta^*$.  Write $\boldvarphi = \boldvarphi^* + \bolddelta$ for
a general unconditional canonical parameter.  The Laplace transform
(which in this case is also a moment generating function)
of $P_{\boldvarphi^*}$ is
\begin{equation} \label{eq:this-is-it}
   \int e^{\inner{\boldx, \bolddelta}} P_{\boldvarphi^*}(d \boldx)
   =
   \exp\bigl( \psi(\boldvarphi^* + \bolddelta) - \psi(\boldvarphi^*) \bigr),
\end{equation}
which agrees with the analysis in \eqref{eq:move}.  Now we define the
full canonical parameter space
$$
   \Phi = \set{ \boldvarphi \in \real^J : \psi(\boldvarphi) < \infty }
$$
(we are using the original parameter $\boldvarphi$ instead of the ``new''
parameter $\bolddelta$).  It stands to reason that $\Phi$ is just the
set of points obtained by mapping $\Theta$ through the change of parameter
defined by \eqref{eq:new}.  The whole point of this appendix is to show
that what seems obvious actually is obvious.

\subsection{Leaf Nodes}

For $j$ a leaf node, the only part of the integral in \eqref{eq:this-is-it}
involving $x_j$ is
$$
   \int e^{x_j \delta_j} P_{\theta_j^*, j}(d x_j \mid x_{p(j)}).
$$
Now from the uniparameter case of the exponential family theory embodied
in \eqref{eq:this-is-it} we see that
$$
   \int e^{x_j \delta_j} P_{\theta_j^*, j}(d x_j \mid x_{p(j)} = 1)
   =
   \exp\bigl( \psi_j(\theta_j^* + \delta_j) - \psi_j(\theta_j^*) \bigr),
$$
from which it follows from the multiplication rule for moment generating
functions and the structure of aster models that
\begin{equation} \label{eq:leaf}
   \int e^{x_j \delta_j} P_{\theta_j^*, j}(d x_j \mid x_{p(j)})
   =
   \exp\bigl( x_{p(j)}
   [ \psi_j(\theta_j^* + \delta_j) - \psi_j(\theta_j^*) ] \bigr).
\end{equation}
This is finite if and only if
$$
   \varphi_j = \theta_j = \theta_j^* + \delta_j
   = \varphi_j^* + \delta_j \in \Theta_j
$$
in short if
$$
   \varphi_j = \theta_j \in \Theta_j.
$$
So that is the finiteness condition for leaf nodes.  The part of the integral
\eqref{eq:this-is-it} that pertains to $\theta_j$ and $\varphi_j$
has the ``obvious'' condition for being finite.

\subsection{Non-Leaf Nodes}

Now consider a non-leaf node $j$ whose children are all leaf nodes.
After integrating out the $x_m$ for all child nodes $m$, as above,
the only part of the integral in \eqref{eq:this-is-it} that contains
$x_j$ is
\begin{subequations}
\begin{equation} \label{eq:phred}
   \int \exp\left( x_j \left[ \delta_j + \sum_{m \in S(j)}
     \bigl( \psi_m(\theta_m) - \psi_m(\theta_m^*) \bigr) \right] \right)
   P_{\theta_j^*, j}(d x_j \mid x_{p(j)})
\end{equation}
where we have written $\theta_m = \varphi_m = \theta_m^* + \delta_m$
for the $m$, all of which are leaf nodes.
We also write $\varphi_j = \varphi_j^* + \delta_j$ and note that from
\eqref{eq:new} we have
$$
   \varphi_j^* + \delta_j + \sum_{m \in S(j)} \psi_m(\theta_m)
   =
   \varphi_j + \sum_{m \in S(j)} \psi_m(\theta_m)
   = \theta_j
$$
and similarly
$$
   \varphi_j^* + \sum_{m \in S(j)} \psi_m(\theta_m^*)
   =
   \theta_j^*
$$
so the term in large square brackets in \eqref{eq:phred}
is just $\theta_j^* - \theta_j$.  Hence \eqref{eq:phred} is
\begin{equation} \label{eq:phred-junior}
   \int \exp\left( x_j \left[ \theta_j^* - \theta_j \right] \right)
   P_{\theta_j^*, j}(d x_j \mid x_{p(j)})
   =
   \exp\bigl( x_{p(j)}
   [ \psi_j(\theta_j) - \psi_j(\theta_j^*) ] \bigr).
\end{equation}
\end{subequations}
and we see that \eqref{eq:phred} and \eqref{eq:phred-junior} say exactly
the same thing as \eqref{eq:leaf}.

However, the implication about finiteness of bits of \eqref{eq:this-is-it}
is a bit different than the case for leaf nodes.  Now we have that
$$
   \theta_j \in \Theta_j
$$
is the finiteness condition for the term on the right hand side of
\eqref{eq:phred-junior} (no surprise there) and that translates to
the following about $\varphi_j$.  Since
\begin{align*}
   \theta_j
   & =
   \varphi_j + \sum_{m \in S(j)} \psi_m(\theta_m)
\end{align*}
the finiteness condition for this section is
\begin{equation} \label{eq:finite}
   \varphi_j \in \Theta_j - \sum_{m \in S(j)} \psi_m(\theta_m).
\end{equation}

\subsection{All Nodes by Mathematical Induction}

Now assume (the induction hypothesis) that
\begin{multline}
   \label{eq:induce}
   \int \exp\left( x_j \left[ \delta_j + \sum_{m \in S(j)}
     \bigl( \psi_m(\theta_m) - \psi_m(\theta_m^*) \bigr) \right] \right)
   P_{\theta_j^*, j}(d x_j \mid x_{p(j)})
   \\
   =
   \exp\bigl( x_{p(j)}
   [ \psi_j(\theta_j) - \psi_j(\theta_j^*) ] \bigr).
\end{multline}
As we have seen, this holds for all of the nodes we have examined so far.
But then we see that by the argument in the preceding section that if
\eqref{eq:induce} holds for all children (successors) of a node, then it
holds for the node itself.

Thus \eqref{eq:finite} is the finiteness condition for all nodes,
including leaf nodes for which $S(j)$ is the empty set.

Hence $\Phi$ is indeed the image of $\Theta$ under the map \eqref{eq:new}.
And the argument did turn out to be ``obvious''.

\subsection{Convexity}

The only non-obvious fact in this whole appendix is that hence $\Phi$ must
be a convex set (like all full canonical parameter spaces).  Moreover,
since the map \eqref{eq:new} is a diffeomorphism (both it and its inverse
are differentiable) between the interiors of $\Theta$ and $\Phi$, it follows
that $\Theta$ and $\Phi$ are both open sets or both not open.

\subsection{Regularity}

The important special case where the full canonical parameter space is an
open set is called \emph{regular} (the family is a \emph{regular} exponential
family) \citep[p.~116]{barndorff}.  They are the most well-behaved with respect
to maximum likelihood (no boundaries of the parameter space to worry about).

Since $\Theta$ is open if and only if each $\Theta_j$ is, we see that
the full flat exponential family (with the unconditional canonical
parameterization) is regular if and only if each one-parameter conditional
family is regular.

\subsection{Steepness}

A full exponential family is \emph{steep} \cite[p.~117]{barndorff} if
given $\boldvarphi_i$ in the interior of $\Phi$ and $\boldvarphi_b$ on the
boundary of $\Phi$, then
$$
   \inner{\nabla \psi\bigl(t \boldvarphi_i + (1 - t) \boldvarphi_b \bigr),
   \boldvarphi_b - \boldvarphi_i} \to \infty,
   \qquad \text{as $t \downarrow 0$}.
$$
Clearly this carries over to any flat subfamily (formed by intersecting
the full canonical parameter space $\Phi$ with an affine subspace).
The subfamily is steep if the full family is.  Thus we concentrate on
the full (FEF) family only.

An equivalent (if and only if) condition for steepness is that the MLE map
$\boldx \mapsto \boldvarphihat(\boldx)$ is one-to-one and is found by
solving the ``likelihood equations'' which have ``observed equals expected''
form, as in equation \eqref{eq:score-uncon} and the preceding unnumbered
equation \citep[Theorem~9.14, Corollary~9.6 and their surrounding
discussion]{barndorff}.

Yet another equivalent (if and only if) condition for steepness is that
the mean value parameterization
map, $\nabla \psi : \boldvarphi \mapsto \boldtau$ in aster model notation,
maps the interior of the full canonical parameter space $\Phi$ onto the
interior of the closed convex support of the canonical statistic
\citep[pp.~117, 142, and 152]{barndorff}.

Let $a_j$ and $b_j$ be the endpoints (possibly infinite) of the full
mean-value parameter space of the one-parameter exponential family
for the $j$-th node.  The endpoints themselves may or may not be in
the mean-value parameter space
$$
   \set{ \psi_j'(\theta) : \theta \in \Theta_j }
$$
but the closed convex support of the $j$-th one-parameter exponential
family is the closed interval
$\real \cap [a_j, b_j]$ if this $j$-th family is steep
(the point of the $\real \cap \vphantom{a_j}$ is
to omit $- \infty$ and $+ \infty$
if either $a_j = - \infty$ or $b_j = + \infty$).

Clearly from equation \eqref{eq:mvp-exp} the closed convex support of
$X_j$ in the FEF is the closed interval
\begin{equation} \label{eq:sally}
   \real \cap
   \left[
   X_{i f(j)}
   \prod_{\substack{m \in J \\ j \preceq m \prec f(j)}} a_m,
   X_{i f(j)}
   \prod_{\substack{m \in J \\ j \preceq m \prec f(j)}} b_m
   \right].
\end{equation}
It is clear also that as $\boldtheta$ runs over
the interior of $\Theta$ the
mean value parameter $\boldtau(\boldtheta)$ runs over
the interior of \eqref{eq:sally}.  Hence if each of the
one-parameter exponential families is steep, then the FEF of an
aster model is steep.

\section{Conclusions}

This whole appendix is (in hindsight) ``obvious.''  What we have learned
is that what we thought was obvious is indeed obvious and there is no
problem with maximum likelihood in an aster model if each of the one-parameter
exponential families is regular or steep.  Moreover, we have learned that
the canonical parameter space of the FEF is convex (like every full canonical
parameter space of every full exponential family) something that is ``obvious''
only from exponential family theory and not from looking at the definition
of the map \eqref{eq:new} between conditional and unconditional canonical
parameterizations.

