
\documentclass[11pt,twoside,notitlepage]{report}

\usepackage{indentfirst}
\usepackage{amsmath}
\usepackage{amssymb}
\usepackage{natbib}
\usepackage{url}
\usepackage{graphicx}
\usepackage[utf8]{inputenc}

\usepackage{Sweave}

\usepackage{ifthen}
\newboolean{techrept}
\setboolean{techrept}{true}
\newboolean{submit}
\setboolean{submit}{false}


\newcommand{\set}[1]{\{\, #1 \,\}}
\newcommand{\inner}[1]{\langle #1 \rangle}

\newcommand{\mdot}{\,\cdot\,}

\RequirePackage{amsmath}

\DeclareMathOperator{\pr}{pr}

\ifthenelse{\boolean{submit}}{
\newcommand{\bolda}{a}
\newcommand{\boldx}{x}

\newcommand{\boldbeta}{\beta}
\newcommand{\bolddelta}{\delta}
\newcommand{\boldtheta}{\theta}
\newcommand{\boldthetahat}{\hat{\theta}}
\newcommand{\boldvarphi}{\varphi}
\newcommand{\boldvarphihat}{\hat{\varphi}}
\newcommand{\boldtau}{\tau}
\newcommand{\boldtauhat}{\hat{\tau}}
\newcommand{\boldxi}{\xi}
\newcommand{\boldzeta}{\zeta}
\newcommand{\boldeta}{\eta}

\newcommand{\boldbetahat}{\hat{\beta}}
\newcommand{\boldetahat}{\hat{\eta}}
\newcommand{\boldzetahat}{\hat{\zeta}}

\newcommand{\boldzero}{0}

\newcommand{\boldA}{A}
\newcommand{\boldG}{G}
\newcommand{\boldM}{M}
\newcommand{\boldMtwiddle}{\widetilde{M}}
\newcommand{\boldX}{X}
\newcommand{\boldY}{Y}

\newcommand{\myPr}{\pr}
}{
\newcommand{\bolda}{\mathbf{a}}
\newcommand{\boldx}{\mathbf{x}}

\newcommand{\boldbeta}{{\boldsymbol{\beta}}}
\newcommand{\bolddelta}{{\boldsymbol{\delta}}}
\newcommand{\boldtheta}{{\boldsymbol{\theta}}}
\newcommand{\boldthetahat}{{\boldsymbol{\hat{\theta}}}}
\newcommand{\boldvarphi}{{\boldsymbol{\varphi}}}
\newcommand{\boldvarphihat}{{\boldsymbol{\hat{\varphi}}}}
\newcommand{\boldtau}{{\boldsymbol{\tau}}}
\newcommand{\boldtauhat}{{\boldsymbol{\hat{\tau}}}}
\newcommand{\boldxi}{{\boldsymbol{\xi}}}
\newcommand{\boldzeta}{{\boldsymbol{\zeta}}}
\newcommand{\boldeta}{{\boldsymbol{\eta}}}

\newcommand{\boldbetahat}{{\boldsymbol{\hat{\beta}}}}
\newcommand{\boldetahat}{{\boldsymbol{\hat{\eta}}}}
\newcommand{\boldzetahat}{{\boldsymbol{\hat{\zeta}}}}

\newcommand{\boldzero}{{\boldsymbol{0}}}

\newcommand{\boldA}{\mathbf{A}}
\newcommand{\boldG}{\mathbf{G}}
\newcommand{\boldM}{\mathbf{M}}
\newcommand{\boldMtwiddle}{\widetilde{\mathbf{M}}}
\newcommand{\boldX}{\mathbf{X}}
\newcommand{\boldY}{\mathbf{Y}}

\newcommand{\myPr}{\Pr}
}

\newcommand{\groupset}{\mathcal{G}}

\DeclareMathOperator{\cov}{cov}
\DeclareMathOperator{\var}{var}
% \DeclareMathOperator{\mrca}{mrca}
% \DeclareMathOperator{\dom}{dom}
% \DeclareMathOperator{\ran}{ran}
\DeclareMathOperator{\logit}{logit}

\RequirePackage{amsfonts}
\newcommand{\real}{\mathbb{R}}



% remove room for headers, add to textheight

  \addtolength{\textheight}{\headheight}
  \addtolength{\textheight}{\headsep}
  \setlength{\headheight}{0 pt}
  \setlength{\headsep}{0 pt}

% adjust so right and left hand pages have different margins
% not sure why these particular numbers were picked or if they
% still make sense

  % \showthe\evensidemargin
  % \showthe\oddsidemargin
  % \showthe\textwidth

  % \evensidemargin  15 pt
  % \oddsidemargin 61.5 pt
  % \textwidth 392.5 pt

  \evensidemargin 28.75 pt
  \oddsidemargin 75.25 pt
  \textwidth 365 pt

\begin{document}

  \vspace*{0.9375in}
  \begin{center}
    {\bfseries Aster Models for Life History Analysis} \\
    By \\
    Charles J. Geyer, Stuart Wagenius, and Ruth G. Shaw \\
    Technical Report No.~644 \\
    School of Statistics \\
    University of Minnesota \\
%       April 20, 2005 \\
%       \today
    September 5, 2005
  \end{center}
  \thispagestyle{empty}
  \cleardoublepage
  \setcounter{page}{1}
  \thispagestyle{empty}


\begin{center} \bfseries Summary \end{center}

We present a new class of statistical models,
designed for life history analysis of plants and animals,
that allow joint analysis of data on survival and reproduction
over multiple years, allow for variables
having different probability distributions, and correctly
account for the dependence of variables on earlier variables.
We illustrate their utility with an analysis of data
taken from an experimental study of \emph{Echinacea angustifolia}
sampled from remnant prairie populations in western Minnesota.
These models generalise both generalised linear models and survival analysis.
The joint distribution is factorised as a product of conditional distributions,
each an exponential family with the conditioning variable being the sample
size of the conditional distribution.
The model may be heterogeneous, each conditional distribution being from
a different exponential family.
We show that the joint distribution is from a flat exponential family and
derive its canonical parameters, Fisher information and other properties.
These models are implemented in an R package `aster' available from
the Comprehensive R Archive Network, CRAN.



\vspace{0.2in}
\sloppypar
\noindent
Keywords:
    Conditional Exponential Family;
    Curved Exponential Family;
    Flat Exponential Family;
    Generalized Linear Model;
    Graphical Model;
 %  Life History Analysis;  %%%% now in title
    Maximum Likelihood;
    Nuisance Variable.

\vspace{0.2in}
\sloppypar
This technical report consists of a draft paper about aster models
supplemented by 5 appendicies on technical subjects.
\begin{itemize}
\item Appendix~\ref{ch:predict} (p.~\pageref{ch:predict})
gives details about ``prediction''
(what the \verb@predict.aster@ function does), although the technical
details are about change-of-parameter formulas and their derivatives.
\item Appendix~\ref{ch:unit} (p.~\pageref{ch:unit})
gives details about the one-parameter exponential families currently
available as conditional distribution of variables given their parent variable
in the \verb@aster@ package.
\item Appendix~\ref{ch:sim} (p.~\pageref{ch:sim})
gives details about simulating a Poisson distribution conditional on being
nonzero (the only non-trivial issue being getting efficient simulation for
unconditional means close to zero).
\item Appendix~\ref{ch:coneflr} (p.~\pageref{ch:coneflr})
gives details of the data analysis that is briefly described
in the draft paper.
\item Appendix~\ref{ch:spaces} (p.~\pageref{ch:spaces})
shows that steepness of conditional exponential families
implies stepness of corresponding unconditional families,
and the full unconditional family gets no new parameter points (that do not
correspond to conditional parameter points).
\end{itemize}

Note added for this git repository.  The text file of the draft paper
that was used to make this technical report seems to have been lost.
(We were not using version control when the paper was written.)
So this document does not reproduce the technical report exactly
(the text here is the text of the published version of the paper).
The PDF of the original technical is found
at \url{https://www.stat.umn.edu/geyer/aster/tr644.pdf}.

  \thispagestyle{empty}
  \cleardoublepage
  \setcounter{page}{1}

\chapter{Draft Paper} \label{ch:aster}


\hyphenation{life-span}

\section{Introduction}

This article introduces a class of statistical models
we call `aster models'.
They were invented for life history analysis
of plants and animals and are best introduced by an example
about perennial plants
observed over several years.  For each individual planted,
at each census, we record
whether or not it is alive, whether or not it flowers, and its number of
flower heads.  These data are complicated, especially when recorded
for several years, but simple conditional models may
suffice.  We consider mortality status, dead or alive, to be Bernoulli
given the preceding mortality status.  Similarly, flowering status
given mortality status is also Bernoulli.
Given flowering, the number of flower heads
may have a zero-truncated Poisson distribution \citep{martin}.
Figure~\ref{fig:graph} % (page~\pageref{fig:graph})
shows the graphical model for a single individual.
% The full graph consists of many copies of this graph, one for each individual.
\begin{figure}
\begin{center}
\setlength{\unitlength}{0.4 in}
\begin{picture}(4,5)(-3,-5)
\put(0,0){\makebox(0,0){$1$}}
\put(-1,-1){\makebox(0,0){$M_1$}}
\put(-2,-2){\makebox(0,0){$M_2$}}
\put(-3,-3){\makebox(0,0){$M_3$}}
\put(0,-2){\makebox(0,0){$F_1$}}
\put(-1,-3){\makebox(0,0){$F_2$}}
\put(-2,-4){\makebox(0,0){$F_3$}}
\put(1,-3){\makebox(0,0){$H_1$}}
\put(0,-4){\makebox(0,0){$H_2$}}
\put(-1,-5){\makebox(0,0){$H_3$}}
% \multiput(-0.75,-0.75)(-1,-1){3}{\vector(1,1){0.5}}
% \multiput(-0.25,-1.75)(1,-1){2}{\vector(-1,1){0.5}}
% \multiput(-1.25,-2.75)(1,-1){2}{\vector(-1,1){0.5}}
% \multiput(-2.25,-3.75)(1,-1){2}{\vector(-1,1){0.5}}
\multiput(-0.25,-0.25)(-1,-1){3}{\vector(-1,-1){0.5}}
\multiput(-0.75,-1.25)(1,-1){2}{\vector(1,-1){0.5}}
\multiput(-1.75,-2.25)(1,-1){2}{\vector(1,-1){0.5}}
\multiput(-2.75,-3.25)(1,-1){2}{\vector(1,-1){0.5}}
\end{picture}
\end{center}
\caption{Graph for \emph{Echinacea} aster data.
Arrows go from parent nodes to child nodes.
Nodes are labelled by their associated variables.
The only root node is associated with the constant variable $1$.
$M_j$ is the mortality status in year $2001 + j$.
$F_j$ is the flowering status in year $2001 + j$.
$H_j$ is the flower head count in year $2001 + j$.
The $M_j$ and $F_j$ are Bernoulli conditional on their parent
variables being one, and zero otherwise.
The $H_j$ are zero-truncated Poisson
conditional on their parent
variables being one, and zero otherwise.}
\label{fig:graph}
\end{figure}
This aster model generalises both
discrete time Cox regression \citep{cox,breslow-diss,breslow}
and generalised linear models \citep{man}.
Aster models apply to any similar conditional modelling.
We could, for example, add other variables, such as seed count
modelled conditional on flower head count.

% The most important advantage of these models is the simplest one.
A simultaneous analysis that models the joint distribution of
all the variables in a life history analysis
can answer questions that cannot be addressed through separate
analyses of each variable conditional on the others.
Joint analysis also deals with structural zeros in the data;
for example, a dead individual remains dead and cannot flower,
so in Fig.~\ref{fig:graph} any arrow that leads from a variable that is zero
to another variable implies that the other variable
must also be zero.
Such zeros
present intractable missing data problems in separate analyses
of individual variables.  Aster models have no problem with structural
zeros; likelihood inference automatically handles them correctly.

Aster models are simple graphical models
\citep[\S~3.2.3]{lauritzen}
in which the joint density is a product of conditionals as in
equation \eqref{eq:product} below.  No knowledge of graphical
model theory is needed to understand aster models.
One innovative aspect of aster models is the interplay between two
parameterisations described
in \S\S~\ref{sec:CEF} and~\ref{sec:FEF} below.
The `conditional canonical parameterisation'
arises when each conditional distribution in the product
is an exponential family and we
use the canonical parameterisation for each.
The `unconditional canonical parameterisation'
arises from observing that the joint model is a full flat exponential family
\citep[Ch.~8]{barndorff} and using the canonical parameters for that
family, defined by equation \eqref{eq:new} below.

\section{Aster Models}

\subsection{Factorisation and graphical model}

Variables in an aster model are denoted by $X_j$, where $j$ runs over the nodes
of a graph.
A general aster model is a chain graph model \citep[pp.~7, 53]{lauritzen}
having both arrows, corresponding to directed edges, and lines, corresponding
to undirected edges.
Figure~\ref{fig:graph} is special, having only arrows.
Arrows go from parent to child, lines between neighbours.
Nodes that are not children are called root nodes.
Those that are not parents are called terminal nodes.

Let $F$ and $J$ denote root and non-root nodes.  Aster models have very
special chain graph structure determined by a partition $\groupset$ of $J$ and
a function $p : \groupset \to J \cup F$.
For each $G \in \groupset$ there is
an arrow from $p(G)$ to each element of $G$ and a line
between each pair of elements of $G$.
For any set $S$, let $\boldX_S$ denote the
vector whose components are $X_j$, $j \in S$.
The graph determines a factorisation
\begin{equation} \label{eq:product}
   \myPr(\boldX_J | \boldX_F)
   =
   \prod_{G \in \groupset} \myPr \{ \boldX_G | X_{p(G)} \};
\end{equation}
compare equation~3.23 in \citet{lauritzen}.

Elements of $\groupset$ are called
chain components
because they are connectivity components of the chain graph
\citep[pp.~6--7]{lauritzen}.
Since Fig.~\ref{fig:graph} has no undirected edge,
each node is a chain component by itself.
Allowing nontrivial chain components
allows the elements of $\boldX_G$ to be conditionally dependent
given $X_{p(G)}$ with merely notational changes to the theory.
In our example in \S~\ref{sec:example}
the graph consists of many copies of Fig.~\ref{fig:graph}, one for each
individual plant.  Individuals have no explicit representation.
For any set $S$, let $p^{-1}(S)$ denote the set of $G$ such that $p(G) \in S$.
Then each subgraph consisting of one $G \in p^{-1}(F)$, its descendants,
children, children of children, etc., and arrows and lines connecting them,
corresponds to one individual.
If we make each such $G$ have a distinct root element $p(G)$, then
the set of descendants of each root node corresponds to one individual.
Although all individuals in our example have the same subgraph, this is
not required.

\subsection{Conditional exponential families} \label{sec:CEF}

We take each of the conditional distributions in \eqref{eq:product}
to be an exponential family having canonical statistic $\boldX_G$ 
that is the sum of $X_{p(G)}$
independent and identically distributed random vectors,
possibly a different such family for each $G$.
Conditionally, $X_{p(G)} = 0$ implies that $\boldX_G = \boldzero_G$ almost surely.
If $j \neq p(G)$ for any $G$, then the
values of $X_j$ are unrestricted.
If the distribution of $\boldX_G$ given $X_{p(G)}$ is infinitely
divisible, such as Poisson or normal, for each $G \in p^{-1}(\{j\})$,
then $X_j$ must be nonnegative and real-valued.
Otherwise, $X_j$ must be nonnegative and integer-valued.

The loglikelihood for the whole family has the form
\begin{equation} \label{eq:logl-zero}
   \sum_{G \in \groupset}
   \left[ \sum_{j \in G} X_j \theta_j - X_{p(G)} \psi_G(\boldtheta_G) \right]
   =
   \sum_{j \in J} X_j \theta_j
   -
   \sum_{G \in \groupset} X_{p(G)} \psi_G(\boldtheta_G),
\end{equation}
where
$\boldtheta_G$ is the canonical parameter vector for the $G$th
conditional family, having components $\theta_j$, $j \in G$,
and $\psi_G$ is the cumulant function for that family
\citep[pp.~105, 139, 150]{barndorff} that satisfies
\begin{align}
   E_{\boldtheta_G} \{ \boldX_G | X_{p(G)} \}
   & =
   X_{p(G)} \nabla \psi_G(\boldtheta_G)
   \label{eq:bartlett-1}
   \\
   \var_{\boldtheta_G} \{ \boldX_G | X_{p(G)} \}
   & =
   X_{p(G)} \nabla^2 \psi_G(\boldtheta_G)
   \label{eq:bartlett-2},
\end{align}
where $\var_\boldtheta(\boldX)$ is
the variance-covariance matrix of $\boldX$
and $\nabla^2 \psi(\boldtheta)$ is
the matrix of second partial derivatives of $\psi$
\citep[p.~150]{barndorff}.

\subsection{Unconditional exponential families} \label{sec:FEF}

Collecting terms with the same $X_j$ in \eqref{eq:logl-zero}, we obtain
\begin{equation*} % \label{eq:star}
   \sum_{j \in J} X_j \left[
   \theta_j
   -
   \sum_{G \in p^{-1}(\{j\})} \psi_G(\boldtheta_G)
   \right]
   -
   \sum_{G \in p^{-1}(F)} X_{p(G)} \psi_G(\boldtheta_G)
\end{equation*}
and see that
\begin{equation} \label{eq:new}
   \varphi_j = \theta_j - \sum_{G \in p^{-1}(\{j\})} \psi_G(\boldtheta_G),
   \qquad j \in J,
\end{equation}
are the canonical parameters of an unconditional exponential family with canonical
statistics $X_j$.  We now write $\boldX$ instead of $\boldX_J$, $\boldvarphi$ instead
of $\boldvarphi_J$, and so forth,
and let $\inner{ \boldX, \boldvarphi }$ denote the inner product $\sum_j X_j \varphi_j$.
Then we can write the loglikelihood of this unconditional family as
\begin{equation} \label{eq:logl-new}
   l(\boldvarphi) = \inner{ \boldX, \boldvarphi } - \psi(\boldvarphi),
\end{equation}
where the cumulant function of this family is
\begin{equation} \label{eq:cumulant-new}
   \psi(\boldvarphi) = \sum_{G \in p^{-1}(F)} X_{p(G)} \psi_G(\boldtheta_G).
\end{equation}
All of the $X_{p(G)}$ in \eqref{eq:cumulant-new}
are at root nodes, and hence are nonrandom,
so that $\psi$ is a deterministic function.  Also
the right-hand side
of \eqref{eq:cumulant-new} is a function
of $\boldvarphi$ by the logic of exponential families
\citep[pp.~105 ff.]{barndorff}.

The system of equations \eqref{eq:new} can be solved for the $\theta_j$
in terms of the $\varphi_j$ in one pass through the equations in any order
that finds $\theta_j$ for children before parents.  Thus \eqref{eq:new}
determines an invertible change of parameter.

\subsection{Canonical affine models} \label{sec:sufficient}

One of the desirable aspects of exponential
family canonical parameter affine models defined by reparameterisation of the form
\begin{equation} \label{eq:M-beta}
   \boldvarphi = a + \boldM \boldbeta,
\end{equation}
where $\bolda$ is a known vector, called the origin,
and $\boldM$ is a known matrix, called the model matrix, is that,
because
$\inner{\boldX, \boldM \boldbeta} = \inner{\boldM^T \boldX, \boldbeta}$,
the result is a new exponential family
with canonical statistic
\begin{equation} \label{eq:sufficient}
   \boldY = \boldM^T \boldX
\end{equation}
and canonical parameter $\boldbeta$.
The dimension of this new family will be the dimension of $\boldbeta$,
if
% the original family was identifiable and
$\boldM$ has full rank.

% If $\boldX$ is a matrix interpreted as a vector, then $\boldM$ is an
% array representing a linear operator $\real^K \to \real^{I \times J}$.
% So \eqref{eq:model-unconditional} means
% \begin{equation} \label{eq:M-beta}
%    \varphi_{i j} = \sum_{k \in K} m_{i j k} \beta_k,
% \end{equation}
% and $\boldY = \boldM^T \boldX$ means
% \begin{equation} \label{eq:sufficient}
%    Y_k
%    =
%    \sum_{i \in I} \sum_{j \in J} X_{i j} m_{i j k}.
% \end{equation}

As is well known \citep[p.~111]{barndorff}, the canonical statistic
of an exponential family is minimal sufficient.
% (if the parameterisation is identifiable).
Since we have both conditional and unconditional
families in play, we stress that this well-known result is about
unconditional families.
A dimension reduction to a low-dimensional sufficient statistic
like \eqref{eq:sufficient} does not occur when the conditional canonical
parameters $\boldtheta$ are modelled affinely,
and this suggests that affine models for the
unconditional parameterisation may be scientifically more interesting
despite their more complicated structure.

\subsection{Mean-value parameters}

Conditional mean values
\begin{equation} \label{eq:mvp-cond}
   \xi_j
   =
   E_{\boldtheta_G} \{ X_j | X_{p(G)} \}
   =
   X_{p(G)} \frac{\partial \psi_G(\boldtheta_G)}{\partial \theta_j},
   \qquad j \in G,
\end{equation}
are not parameters because they contain random data $X_{p(G)}$,
although they play the role of mean-value parameters when
we condition on $X_{p(G)}$, treating it as constant.
By standard exponential family theory \citep[p.~121]{barndorff},
$\nabla \psi_G$ % OOPS! : \boldtheta_G \mapsto \boldxi_G$
is an invertible change of parameter.

Unconditional mean-value parameters are the unconditional expectations
\begin{equation} \label{eq:mvp}
   \boldtau
   =
   E_\boldvarphi \{ \boldX \}
   =
   \nabla \psi(\boldvarphi).
\end{equation}
By standard theory, $\nabla \psi : \boldvarphi \mapsto \boldtau$
is an invertible change of parameter.
The unconditional expectation in \eqref{eq:mvp} can be calculated
using the iterated expectation theorem
\begin{equation} \label{eq:mvp-exp}
   E_\boldvarphi \{ X_j \}
   =
   E_\boldvarphi \{ X_{p(G)} \}
   \frac{\partial \psi_G(\boldtheta_G)}{\partial \theta_j},
   \qquad j \in G,
\end{equation}
where $\boldtheta$ is determined from $\boldvarphi$ by solving
\eqref{eq:new}.
The system of equations \eqref{eq:mvp-exp} can produce the $\tau_j$
in one pass through the equations in any order
that finds $\tau_j$ for parents before children.

\section{Likelihood Inference}

\subsection{Conditional models} \label{sec:CEFth}

% \subsubsection{Score}

The score $\nabla l(\boldtheta)$ for conditional canonical parameters
is particularly simple, having components
\begin{equation} \label{eq:score-saturated-conditional}
   \frac{\partial l(\boldtheta)}{\partial \theta_j}
   =
   X_j - X_{p(G)} \frac{\psi_G(\boldtheta_G)}{\partial \theta_j},
   \qquad j \in G,
\end{equation}
and, if these parameters are modelled affinely as in
\eqref{eq:M-beta} but with $\boldvarphi$ replaced by $\boldtheta$,
then
\begin{equation} \label{eq:score}
   \nabla l(\boldbeta)
   =
   \nabla l(\boldtheta) \boldM.
\end{equation}

The observed Fisher information matrix for $\boldtheta$, which is the matrix
$- \nabla^2 l(\boldtheta)$, is block diagonal with
\begin{equation} \label{eq:cond-obs-fish-diag}
   - \frac{\partial^2 l(\boldtheta)}{\partial \theta_i \partial \theta_j}
   =
   X_{p(G)} \frac{\partial^2 \psi_G(\boldtheta_G)}{\partial \theta_i \partial \theta_j},
   \qquad i, j \in G,
\end{equation}
the only nonzero entries.
The expected Fisher information matrix for $\boldtheta$ is the unconditional expectation
of the observed Fisher information matrix, calculated using
\eqref{eq:cond-obs-fish-diag} and \eqref{eq:mvp-exp}.
If $I(\boldtheta)$ denotes either the observed or the expected Fisher information matrix
for $\boldtheta$ and similarly for other parameters, then
\begin{equation} \label{eq:cond-obs-fish}
   I(\boldbeta)
   =
   \boldM^T I(\boldtheta) \boldM.
\end{equation}

\subsection{Unconditional models} \label{sec:FEFth}

% \subsubsection{Score}

The score $\nabla l(\boldvarphi)$ for unconditional canonical parameters is,
as in every unconditional exponential family, `observed minus expected':
$$
   \frac{\partial l(\boldvarphi)}{\partial \varphi_j}
   =
   X_j - E_\boldvarphi \{ X_j \},
$$
the unconditional expectation on the right-hand side being evaluated
by using \eqref{eq:mvp-exp}.
If these parameters are modelled affinely as in
\eqref{eq:M-beta},
then we have \eqref{eq:score} with $\boldtheta$ replaced by $\boldvarphi$.
Note that \eqref{eq:score-saturated-conditional} is not
`observed minus conditionally expected' if considered as a vector equation
because the conditioning would differ amongst components.

\ifthenelse{\boolean{submit}}{}{\begin{sloppypar}}
Second derivatives with respect to unconditional canonical
parameters of an exponential family are nonrandom, and hence
observed and expected Fisher information matrices $I(\boldvarphi)$ are equal,
given by either of the
expressions $\nabla^2 \psi(\boldvarphi)$ and $\var_\boldvarphi(\boldX)$.
Fix $\boldtheta$ and $\boldvarphi$ related by \eqref{eq:new}.
For $i, j \in G$, the iterated covariance formula gives
\begin{equation} \label{eq:fish-food}
   \cov_\boldvarphi \{ X_i, X_j \}
   =
   \frac{\partial^2 \psi_G(\boldtheta_G)}{\partial \theta_i \partial \theta_j}
   E_\boldvarphi \bigl\{ X_{p(G)} \bigr\}
   +
   \frac{\partial \psi_G(\boldtheta_G)}{\partial \theta_i}
   \frac{\partial \psi_G(\boldtheta_G)}{\partial \theta_j}
   \var_\boldvarphi \bigl\{ X_{p(G)} \bigr\}.
\end{equation}
Otherwise we may assume that $j \in G$ and $i$ is not a descendant of $j$ so that
$\cov_\boldvarphi \{ X_i, X_j | X_{p(G)} \} = 0$
because $X_{j}$ is conditionally independent given $X_{p(G)}$ of
all variables except itself and its descendants.
Then the iterated covariance formula gives
\begin{equation} \label{eq:fish-hook}
   \cov_\boldvarphi \{ X_i, X_j \}
   =
   \frac{\partial \psi_G(\boldtheta_G)}{\partial \theta_j}
   \cov_\boldvarphi \bigl\{ X_i, X_{p(G)} \bigr\}.
\end{equation}
Expectations having been calculated using \eqref{eq:mvp-exp},
variances, the case $i = j$ in \eqref{eq:fish-food}, are calculated
in one pass through \eqref{eq:fish-food} in any order that deals with parents
before children.  Then another pass using \eqref{eq:fish-food} and \eqref{eq:fish-hook} gives covariances.
The information matrix for $\boldbeta$ is given by
\eqref{eq:cond-obs-fish} with $\boldtheta$ replaced by $\boldvarphi$.
\ifthenelse{\boolean{submit}}{}{\end{sloppypar}}

\subsection{Prediction} \label{sec:predict}

By `prediction', we only mean evaluation of a function of estimated
parameters, what the \texttt{predict} function in R does for
generalised linear models.  In aster models we have five different
parameterisations of interest, $\boldbeta$, $\boldtheta$, $\boldvarphi$,
$\boldxi$ and $\boldtau$.  The Fisher information matrix for $\boldbeta$, already
described, handles predictions of $\boldbeta$, so
this section is about `predicting' the remaining four.

One often predicts for new
individuals having different covariate values from the observed individuals.
Then the model matrix $\boldMtwiddle$ used for the prediction is different from
that used for calculating $\boldbetahat$ and the Fisher
information matrix $I(\boldbetahat)$, either observed or expected.

Let $\boldeta$ be the affine predictor, i.~e.,
$\boldeta = \boldtheta$ for conditional models and
$\boldeta = \boldvarphi$ for unconditional models,
let $\boldzeta$ be any one of $\boldtheta$, $\boldvarphi$,
$\boldxi$ and $\boldtau$, and let $f : \boldeta \mapsto \boldzeta$.
Suppose we wish to predict
\begin{equation} \label{eq:todo}
    g(\boldbeta)
    =
    h(\boldzeta)
    =
    h\bigl\{ f(\boldMtwiddle \boldbeta) \bigr\}.
\end{equation}
Then, by the chain rule, \eqref{eq:todo} has derivative
\begin{equation} \label{eq:todo-chain}
    \nabla g(\boldbeta)
    =
    \nabla h(\boldzeta) \nabla f(\boldeta)
    \boldMtwiddle,
\end{equation}
and, by the `usual' asymptotics of maximum likelihood and the delta method,
the asymptotic distribution of
the prediction $h(\boldzetahat) = g(\boldbetahat)$ is
$$
   N\bigl[g(\boldbeta),
   \{\nabla g(\boldbetahat)\} I(\boldbetahat)^{-1} \{\nabla g(\boldbetahat)\}^T
   \bigr],
$$
where $\nabla g(\boldbetahat)$ is given by \eqref{eq:todo-chain}
with $\boldetahat = \boldMtwiddle \boldbetahat$ plugged in for $\boldeta$
and $\boldzetahat = f(\boldetahat)$
plugged in for $\boldzeta$.
We write `predictions' in this complicated form to
separate the parts of the specification,
the functions $h$ and $\nabla h$ and the model matrix $\boldMtwiddle$,
that change from application to application from the part
$\nabla f$ that does not change and can
be dealt with by computer; see
\ifthenelse{\boolean{techrept}}
{Appendix~\ref{ch:predict}}
{Appendix~\ref{tr:ch:predict} of
the technical report at \url{http://www.stat.umn.edu/geyer/aster}}
for details.

To predict mean-value parameters one
must specify new `response' data $X_j$ as well as new `covariate'
data in $\boldMtwiddle$.
Unconditional mean-value parameters $\boldtau$ depend on
$X_j$, $j \in F$, whereas
conditional mean-value parameters $\boldxi$ depend on
$X_j$, $j \in J \cup F$.
It is often interesting, however, to predict $\boldxi$ for hypothetical
individuals with $X_j = 1$ for all $j$, thus obtaining conditional mean
values per unit sample size.

\section{Software} \label{sec:soft}

We have released an R
\citep{rcore} package \texttt{aster} that fits, tests, predicts
and simulates aster
models.  It uses the R formula mini-language, originally developed for
\textsc{genstat} and S \citep{war,cha} so that model fitting is much like
that for linear or
generalised linear models.  The R function \texttt{summary.aster}
provides regression coefficients with standard errors, $z$ statistics
and $p$-values; \texttt{anova.aster} provides likelihood ratio tests
for model comparison; \texttt{predict.aster} provides
the predictions with standard errors described in \S~\ref{sec:predict};
\texttt{raster} generates random aster model data for
simulation studies or parametric bootstrap calculations.
The package is available from CRAN (\url{http://www.cran.r-project.org})
and is open source.

The current version of the package limits
the general model described in this article in several ways.
In predictions, only linear $h$ are allowed in \eqref{eq:todo},
but this can be worked around.
For general $h$, observe that $h(\boldzetahat)$ and $\boldA^T \boldzetahat$,
where $\boldA = \nabla h(\boldzeta)$, have the same standard errors.
Thus, obtain $h(\boldzetahat)$ by one call to \verb@predict.aster@
and the standard errors for $\boldA^T \boldzetahat$,
where $\boldA = \nabla h(\boldzetahat)$ by a second call.
In models, the only conditional families currently implemented are
Bernoulli, Poisson, $k$-truncated Poisson, negative binomial
and $k$-truncated negative binomial, but
adding another one-parameter exponential family would require only
implementation of its $\psi$, $\psi'$ and $\psi''$ functions
and its random variate generator.
Multiparameter conditional families, chain components, are not yet implemented.
Allowing terminal nodes that are two-parameter normal
or allowing child nodes that are multinomial given a common parent
would require more extensive changes to the package.

\section{Example} \label{sec:example}

Data were collected on 570 individuals of \emph{Echinacea angustifolia},
each having the data structure shown in Fig.~\ref{fig:graph}.
These plants were sampled as seeds from seven remnant populations 
that are surviving fragments of the tall-grass prairie that a century
ago covered
western Minnesota and other parts of the Great Plains of North America.
The plants were experimentally randomised 
at the time of planting into a field within 6.5 km of all populations
of origin.  The dataset contains three predictor variables:
\verb@ewloc@ and \verb@nsloc@ give
east-west and north-south positions of individuals
and \verb@pop@ gives their remnant population of origin.
To use the R formula mini-language we need to create
more variables: \verb@resp@ is a vector comprising
the response variables, the $M_j$, $F_j$ and $H_j$;
\verb@level@ is categorical naming the
type of response variable, with values $M$, $F$ and $H$;
\verb@year@ is categorical giving the year;
\verb@varb@ is the interaction of \verb@level@ and \verb@year@;
and \verb@hdct@ is short for $\texttt{level} = H$.

We fitted many models; see
\ifthenelse{\boolean{techrept}}
{Appendix~\ref{ch:coneflr}}
{Appendix~\ref{tr:ch:coneflr} of
the technical report at \url{http://www.stat.umn.edu/geyer/aster}}
for details.
Scientific interest focuses on the model comparison
shown in Table~\ref{tab:anova}. % (page~\pageref{tab:anova}).
\begin{table}
\caption{Tests for model comparison.  The model formulae are given above
and the analysis of deviance below; deviance is double the loglikelihood.}
\label{tab:anova}
\newcommand{\Z}{\hphantom{0}}
\begin{center}
\begin{tabular}{l}
Model 1: \verb@resp ~ varb + level:(nsloc + ewloc)@ \\
Model 2: \verb@resp ~ varb + level:(nsloc + ewloc) + hdct * pop - pop@ \\
Model 3: \verb@resp ~ varb + level:(nsloc + ewloc) + hdct * pop@ \\
Model 4: \verb@resp ~ varb + level:(nsloc + ewloc) + level * pop@
\end{tabular}
\end{center}
\begin{minipage}[t]{\textwidth}
\begin{center}
\begin{tabular}{cccccc}
  Model  & Model & Model    & Test  & Test     & Test \\
  Number & d.~f. & Deviance & d.~f. & Deviance & $p$-value \\
\hline
1  &    15 & 2728.72 &    &       &                \\
2  &    21 & 2712.54 & 6  & 16.18 &     0.013\Z\Z  \\
3  &    27 & 2684.86 & 6  & 27.67 &     0.00011    \\
4  &    33 & 2674.70 & 6  & 10.17 &     0.12\Z\Z\Z
\end{tabular}
\end{center}
d.~f., degrees of freedom.
\end{minipage}
\end{table}
The models are nested.  The affine predictor for Model~4 can be written
\begin{equation} \label{eq:linpred}
    \varphi_j = \mu_{L_j, Y_j} + \alpha_{L_j} U_j + \beta_{L_j} V_j
    + \gamma_{L_j, P_j},
\end{equation}
where $L_j$, $Y_j$, $U_j$, $V_j$ and $P_j$ are \verb@level@, \verb@year@,
\verb@ewloc@, \verb@nsloc@ and \verb@pop@, respectively,
and the alphas, betas and gammas are regression coefficients.
Model~3 is the submodel of Model~4 that imposes the constraint
$\gamma_{M, P} = \gamma_{F, P}$, for all populations $P$.
Model~2 is the submodel of Model~3 that imposes the constraint
$\gamma_{M, P} = \gamma_{F, P} = 0$, for all $P$.
Model~1 is the submodel of Model~2 that imposes the constraint
$\gamma_{L, P} = 0$, for all $L$ and $P$.

All models contain the graph node effect, \verb@varb@ or $\mu_{L_j, Y_j}$,
and the quantitative spatial effect, \verb@level:(nsloc + ewloc)@
or $\alpha_{L_j} U_j + \beta_{L_j} V_j$, which was chosen by
comparing many models; for details see
\ifthenelse{\boolean{techrept}}
{Appendix~\ref{ch:coneflr}.}
{the technical report.}
We explain here only differences amongst models, which involve only
categorical predictors.
In an unconditional aster model, which these are, such terms
require the maximum likelihood estimates of mean-value parameters for each category,
summed over all individuals in the category, to match their observed
values: `observed equals expected'.
Model~4 makes observed equal expected
for total head count $\sum_j H_j$,
for total flowering $\sum_j F_j$, and
for total survival $\sum_j M_j$
within each population.
Model~3 makes observed equal expected
for total head count $\sum_j H_j$ and
for total non-head count $\sum_j (M_j + F_j)$
within each population.
Model~2 makes observed equal expected
for total head count $\sum_j H_j$
within each population.

From purely statistical considerations, Model~3 is the best of these
four nested models.  Model~4 does not fit significantly better.
Model~2 fits significantly worse.
It is difficult to interpret Model~3 scientifically,
because
`non-head count' $\sum_j (M_j + F_j)$ is unnatural, in effect scoring 0 for dead,
1 for alive without flowers or 2 for alive with flowers.

Model~2 is the model of primary scientific interest.
Evolutionary biologists are fundamentally interested in fitness,
but it is notoriously diffcult to measure;
see \citet{beatty}, \citet{keller} and \citet{paul}.
The fitness of an individual may be defined as its
lifetime contribution of progeny to the next generation.
For these
data the most direct surrogate measure for fitness is
total head count $\sum_j H_j$.
The currently available data represent a small fraction of this plant's
lifespan.  To obtain more complete measures of fitness, we are continuing
these experiments and collecting these data for successive years.

Biologists call all our measured variables,
the $M_j$, $F_j$ and $H_j$, `components of fitness'.
Since $M_j$ and $F_j$ contribute to fitness only through $H_j$,
in an aster model the unconditional expectation of $H_j$,
its unconditional mean-value parameter, completely accounts
for the contributions
of $M_j$ and $F_j$.  Strictly speaking, this is not quite true, since
we do not have $H_j$ measured over the whole life span, so the last $M_j$
contains the information about future reproduction,
but it becomes more true as more data are collected in future years.
Moreover, we have no data about life span and do not wish to inject
subjective opinion about future flower head count into the analysis.

The statistical point of this is that
the $M_j$ and $F_j$
are in the model only to produce the correct stochastic structure.
If we could directly model the marginal distribution of the
$H_j$, but we cannot, we would not need the other variables.
They are `nuisance variables' that must be in the
model but are of no interest in this particular analysis;
the mean-value
parameters for those variables are nuisance parameters.
% Statisticians seem not to have studied this kind of nuisance variable
% (there is an established usage meaning `confounded with treatment effect'
% that is not what we mean).
Model~3 is the best according to the likelihood
ratio test, but does it fit the variables of interest better than Model~2?
We do not know of an established methodology for addressing this issue, so
we propose looking at confidence intervals for the mean-value parameters
for total flower head count
shown in Fig.~\ref{fig:plot}. % (page~\pageref{fig:plot}).
\begin{figure}
\begin{center}
  \includegraphics[width=4.5in,keepaspectratio]{coneflr-fig-app}
\end{center}
\caption{Confidence intervals for total head count.  Shown are 95\% non-simultaneous
confidence intervals for the
unconditional expectation of total flower head count, all three years,
for individuals from different
populations and central spatial location.
Solid bars are based on Model~2 in Table~\ref{tab:anova},
and dashed bars are based on Model~3 in Table~\ref{tab:anova}.}
\label{fig:plot}
\end{figure}
Although we have no formal test to propose, we claim that it is obvious
that Model~3 is no better than Model~2 at `predicting' the best surrogate
of expected fitness.  We take this as justification for using Model~2 in
scientific discussion and infer from it significant differences
among the populations in flower head count and, thus, fitness.

It being difficult to interpret Model~3 scientifically, Model~4 is
the next larger readily interpretable model.
Model~4 does fit significantly better than Model~2, $P = 0.00016$, which
implies differences among populations
in mortality and flowering (the $M_j$ and $F_j$) that may be of scientific
interest even though they make no direct contribution to fitness,
since Model~2 already fully accounts for their contributions through $H_j$.

Note that we would have obtained very different results had we used
a conditional model, not shown, see
\ifthenelse{\boolean{techrept}}
{Section~\ref{sec:conditional}.}
{\S~\ref{tr:sec:conditional} of the technical report.}
The parameters of interest are unconditional expectations
of total flower head count.  This alone suggests an unconditional
model.  Furthermore, we see in \eqref{eq:new} that unconditional
aster models `mix levels' passing information up from children to parents.
This is why Model~2 in our example was successful in predicting total head
count while only modelling \verb@pop@ effects at head count nodes.
Since a conditional aster model does not mix levels in this way,
it must model all levels and so usually needs many more parameters
than an unconditional model.

\section{Discussion}

The key idea of aster models, as we see it, is the usefulness of
what we have called unconditional aster models, which have
low-dimensional sufficient statistics \eqref{eq:sufficient}.
Following \citet{gat}, who argued in favour of exponential family
models with sufficient statistics chosen to be
scientifically interpretable, an idea they attributed to \citet{jaynes},
we emphasise the value of these models in analyses of life histories
and overall fitness.

We do not insist, however, the R package \verb@aster@ is even-handed
with respect to conditional and unconditional models and conditional
and unconditional parameters.  Users may use whatever seems best to them.
Any joint analysis is better than any separate
analyses of different variables.
We have, however, one warning.
In an unconditional aster model,
as in all exponential family models, the map from canonical
parameters to mean-value parameters is monotone;
with sufficient statistic vector $\boldY$ given by \eqref{eq:sufficient}
having components $Y_k$ we have
\begin{equation}
\label{eq:modelfoo}
   - \frac{\partial^2 l(\boldbeta)}{\partial \beta_k^2}
   =
   \frac{\partial E_\boldbeta \{ Y_k \}}{\partial \beta_k}
   >
   0.
\end{equation}
This gives regression coefficients
their simple interpretation: an increase in $\beta_k$ causes an increase
in $E_\beta \{ Y_k \}$, other betas being held constant.
The analogue of \eqref{eq:modelfoo} for a conditional model is
\begin{equation}
\label{eq:modelgoo}
   - \frac{\partial^2 l(\boldbeta)}{\partial \beta_k^2}
   =
   \frac{\partial}{\partial \beta_k}
   \sum_{G \in \groupset}
   \sum_{j \in G}
   E_\boldbeta \{ X_j | X_{p(G)} \}
   m_{j k} > 0,
\end{equation}
where $m_{j k}$ are components of $\boldM$.
Because we cannot move the sums in \eqref{eq:modelgoo}
inside the conditional expectation,
there is no corresponding simple interpretation of regression coefficients.
Conditional aster models are therefore algebraically simple but
statistically complicated.
Unconditional aster models are algebraically complicated but statistically
simple.  They can be simply explained as flat
exponential families having the desired sufficient statistics.  They are
the aster models that behave according to the intuitions derived
from linear and generalised linear models.

We saw in our example that aster models allowed us to model fitness
successfully.
In medical applications, Darwinian fitness is rarely interesting,
but aster models may allow data on mortality or survival to be analyzed
in combination with other data, such as quality of life measures.

\section*{Acknowledgements}

Our key equation \eqref{eq:new} was derived about
1980 when R.~G.~S.\ was a graduate student and C.~J.~G.\ not yet
a graduate student.  R.~G.~S.'s Ph.~D.\ adviser Janis Antonovics
provided some funding for development, from a now forgotten source, but
which we thank nevertheless.
High quality life history data, generated recently by S.~W., Julie Etterson,
David Heiser and Stacey Halpern, and work of Helen Hangelbroek,
supported by the U.~S.\ National Science
\ifthenelse{\boolean{submit}}
{Foundation,}
{Foundation (DMS-0083468),}
about what in hindsight we would call
a conditional aster model analysis of mortality data,
motivated implementation of these ideas. 
Were it not for Janis's enthusiastic encouragement back then,
we might not have been able to do this now.  We should also acknowledge
the R team.  Without R we would never have made software as usable and
powerful as the \verb@aster@ package.  We also thank Robert Gentleman
for pointing out the medical implications.

%%%% Note: Biometrika used to have a policy against grant numbers.
%%%% They may cut it.  Just a warning.  They may have changed the policy.

\begin{thebibliography}{}

\bibitem[Barndorff-Nielsen(1978)]{barndorff}
\textsc{Barndorff-Nielsen, O.~E.} (1978).
\newblock \emph{Information and Exponential Families}.
\newblock Chichester: John Wiley.

\bibitem[Beatty(1992)]{beatty}
\textsc{Beatty, J.} (1992).
\newblock Fitness: Theoretical contexts.
\newblock In \emph{Keywords in Evolutionary Biology}, Ed. E.~F. Keller
    and E.~A. Lloyd, pp. 115--9.
\newblock Cambridge, MA: Harvard University Press.

\bibitem[Breslow(1972)]{breslow-diss}
\textsc{Breslow, N.} (1972).
\newblock Discussion of the paper by Cox (1972).
\newblock \emph{J. R. Statist. Soc.} B
    \textbf{34}, 216--7.

\bibitem[Breslow(1974)]{breslow}
\textsc{Breslow, N.} (1974).
\newblock Covariance analysis of censored survival data. 
\newblock \emph{Biometrics}, \textbf{30}, 89-99.

\bibitem[Chambers \& Hastie(1992)]{cha}
\textsc{Chambers, J.~M. \& Hastie, T.~J.} (1992).
\newblock Statistical models.
\newblock In \emph{Statistical Models in S}, Ed. J.~M. Chambers
    and T.~J. Hastie, pp. 13--44.
\newblock Pacific Grove, CA: Wadsworth \& Brooks/Cole.

\bibitem[Cox(1972)]{cox}
\textsc{Cox, D.~R.} (1972).
\newblock Regression models and life-tables (with Discussion).
\newblock \emph{J. R. Statist. Soc.} B
    \textbf{34}, 187--220.

\bibitem[Geyer \& Thompson(1992)]{gat}
\textsc{Geyer, C.~J. \& Thompson, E.~A.} (1992).
\newblock Constrained Monte Carlo maximum likelihood for dependent data
    (with Discussion).
\newblock \emph{J. R. Statist. Soc.} B \textbf{54} 657--99.

\bibitem[Jaynes(1978)]{jaynes}
\textsc{Jaynes, E.~T.} (1978).
\newblock Where do we stand on maximum entropy?
\newblock In \emph{The Maximum Entropy Formalism}, Ed. R.~D. Levine and
    M. Tribus, pp. 15–-118.
\newblock Cambridge, MA: MIT Press.

\bibitem[Keller(1992)]{keller}
\textsc{Keller, E.~F.} (1992).
\newblock Fitness: Reproductive ambiguities.
\newblock In \emph{Keywords in Evolutionary Biology}, Ed. E.~F. Keller
    and E.~A. Lloyd, pp. 120--1.
\newblock Cambridge, MA: Harvard University Press.

\bibitem[Lauritzen(1996)]{lauritzen}
\textsc{Lauritzen, S.~L.} (1996).
\newblock \emph{Graphical Models}.
\newblock New York: Oxford University Press.

\bibitem[Martin, et al.(2005)Martin, Wintle, Rhodes, Kuhnert, Field, Low-Choy,
    Tyre \& Possingham]{martin}
\textsc{Martin, T.~G.,
Wintle, B.~A.,
Rhodes, J.~R.,
Kuhnert, P.~M.,
Field, S.~A.,
Low-Choy, S.~J.,
Tyre, A.~J.
\&
Possingham, H.~P.}
(2005).
\newblock Zero tolerance ecology: improving ecological inference
    by modelling the source of zero observations.
\newblock \emph{Ecol. Lett.} \textbf{8} 1235--46.

% \bibitem[McCullagh(1980)]{pmcc}
% \textsc{McCullagh, P.} (1980).
% \newblock Regression models for ordinal data (with Discussion).
% \newblock \emph{J. R. Statist. Soc.} B \textbf{42} 109--42. 

\bibitem[McCullagh \& Nelder(1989)]{man}
\textsc{McCullagh, P. \& Nelder, J.~A.} (1989).
\newblock \emph{Generalized Linear Models}, 2nd ed.
\newblock London: Chapman and Hall.

\bibitem[Paul(1992)]{paul}
\textsc{Paul, D.} (1992).
\newblock Fitness: Historical perspective.
\newblock In \emph{Keywords in Evolutionary Biology}, Ed. E.~F. Keller
    and E.~A. Lloyd, pp. 112--4.
\newblock Cambridge, MA: Harvard University Press.

\bibitem[R Development Core Team(2004)]{rcore}
\textsc{R Development Core Team} (2004).
\newblock R: A language and environment for statistical computing.
\newblock R Foundation for Statistical Computing, Vienna, Austria.
\newblock \url{http://www.R-project.org}.

\bibitem[Wilkinson \& Rogers(1973)]{war}
\textsc{Wilkinson, G.~N. \& Rogers, C.~E.} (1973).
\newblock Symbolic description of factorial models for analysis of variance.
\newblock \emph{Appl. Statist.} \textbf{22}, 392--9.

\end{thebibliography}

% \begin{center} \LARGE REVISED DOWN TO HERE \end{center}



\appendix

\chapter{More Details on Prediction} \label{ch:predict}


\section{Introduction}

\subsection{Functions of Regression Coefficients}

Suppose we have an aster model, a maximum likelihood estimate
$\boldbetahat$, and Fisher information $I(\boldbetahat)$,
whether observed or expected makes no difference.  The asymptotic
distribution of $\boldbetahat$ is
$$
   \text{Normal}\bigl(\boldbeta, I(\boldbetahat)^{-1} \bigr)
$$
where $\boldbeta$ is the true unknown parameter value.

But we usually don't want to make predictions about $\boldbeta$
(regression coefficients are meaningless, only probabilities and expectations
are directly interpretable) but about some function
$g(\boldbeta)$, where $g$ is a scalar-valued or vector-valued function.
The delta method then says that the asymptotic distribution of
$g(\boldbetahat)$ is
$$
   \text{Normal}\bigl(g(\boldbeta),
   [\nabla g(\boldbetahat)] I(\boldbetahat)^{-1} [\nabla g(\boldbetahat)]^T
   \bigr)
$$
Since we already have $\boldbetahat$ and $I(\boldbetahat)$,
we only need $\nabla g(\boldbetahat)$ to finish our problem.

\subsection{Functions of Other Parameters}

Since regression coefficients are meaningless, users will typically
want to specify a function of some other parameter, for example the
unconditional mean value parameter $\boldtau$, which, of course,
is itself a function of $\boldbeta$ if the model
is correct.  That is, we have a composition
$$
   g = h \circ f_{\boldbeta, \boldtau}
$$
where $h$ is an arbitrary function specified by the user
and $f_{\boldbeta, \boldtau}$ is the map $\boldbeta \mapsto \boldtau$.
The point is that $f_{\boldbeta, \boldtau}$ is quite complicated
but is understood (or should be understood) by the computer,
whereas $h$ may be quite simple.

Of course, from the chain rule we have
\begin{subequations}
\begin{equation} \label{eq:chain-one}
   \nabla g(\boldbeta)
   = 
   \nabla h(\boldtau) \circ \nabla f_{\boldbeta, \boldtau}(\boldbeta)
\end{equation}
Since $\nabla h(\boldtau)$ and $\nabla f_{\boldbeta, \boldtau}(\boldbeta)$
are linear operators represented by matrices (of partial derivatives),
the composition here is represented by matrix multiplication.
Of course, we want to use this with estimates plugged in.
\begin{equation} \label{eq:chain-two}
   \nabla g(\boldbetahat)
   = 
   \nabla h(\boldtauhat) \circ \nabla f_{\boldbeta, \boldtau}(\boldbetahat)
\end{equation}
So the general idea is that the user supplies the easy part,
the matrix representing $\nabla h(\boldtauhat)$, and the computer does
the hard part, the matrix
representing $\nabla f_{\boldbeta, \boldtau}(\boldbetahat)$.

There is one slight complication.  The user often provides part of the
specification of $\nabla f_{\boldbeta, \boldtau}(\boldbetahat)$,
for which see Section~\ref{sec:covariate} below.

\subsection{What Other Parameters?}

We suppose that in place of $\boldtau$ in the preceding section, the
user may chose to specify $h$ in terms of any of the parameters we use
in discussing aster models.  There are four (the one already
mentioned and three others)
\begin{itemize}
\item $\boldtheta$, the conditional canonical parameters.
\item $\boldvarphi$, the unconditional canonical parameters.
\item $\boldxi$, the conditional mean value parameters.
\item $\boldtau$, the unconditional mean value parameters.
\end{itemize}
Letting $\boldzeta$ stand for any one of these parameters,
we replace $\boldtau$ by $\boldzeta$ in \eqref{eq:chain-two}
obtaining
\begin{equation} \label{eq:chain-three}
   \nabla g(\boldbetahat)
   = 
   \nabla h(\boldzetahat) \circ \nabla f_{\boldbeta, \boldzeta}(\boldbetahat)
\end{equation}
we need the user to be able to specify a $\nabla h(\boldzetahat)$ and
have the computer produce the required
$\nabla f_{\boldbeta, \boldzeta}(\boldbetahat)$.

\subsection{What Covariates?} \label{sec:covariate}

An important thing the \verb@predict.lm@ function in R does is allow
prediction at covariate values other than those in the observed data.
In fact, this is its main ``feature.''  The multiplicity of parameter
values found in aster models is absent in simple least squares
regression.

Let $\boldeta$ be the canonical parameter that is linearly modeled in
terms of $\boldbeta$ (either $\boldtheta$ for a conditional model
or $\boldvarphi$ for an unconditional model).  Then the ``generalized linear''
part of the aster model is
$$
   \boldeta = \boldM \boldbeta
$$
where $\boldM$ is a linear operator, represented by a matrix
(the model matrix), but, \emph{and this is important}, the $\boldM$
used in the prediction problem need not be the $\boldM$ used in fitting
the model to observed data.  The two model matrices must have the same
column dimension and the columns must have the same meaning (covariate
variables), but the rows need bear no analogy.  Each row of the model
matrix for the real data corresponds to a real individual, but rows of
the model matrix for a prediction problem may correspond to entirely
hypothetical individuals, or newly observed individuals not in the
original data, or whatever.  In this note
we use the notation $\boldM$
to stand for the model matrix involved in our prediction problem.
The model matrix for the original data, when needed, will be
denoted $\boldM_{\text{orig}}$.

Now we can write our function to predict as the composition
$$
   g = h \circ f_{\boldeta, \boldzeta} \circ \boldM
$$
and the chain rule with plug-in becomes
\begin{equation} \label{eq:chain-four}
   \nabla g(\boldbetahat)
   = 
   \nabla h(\boldzetahat) \circ \nabla f_{\boldeta, \boldzeta}(\boldetahat)
   \circ \boldM
\end{equation}
\end{subequations}
(the derivative of a linear operator being the operator itself).

So the division of labor we envisage is that the user will specify the
two matrices $\nabla h(\boldzetahat)$ and $\boldM$ (the latter either
explicitly or, more usually, implicitly
by specifying a new data frame to be used with the formula for the regression)
and the computer must figure out $\nabla f_{\boldeta, \boldzeta}(\boldetahat)$
by itself.

\section{Changes of Parameter}

\subsection{Identity}

For two ``changes of parameter'' of interest to us,
$f_{\boldeta, \boldzeta}$ is the identity mapping and hence so is
$\nabla f_{\boldeta, \boldzeta}$.
These are the cases $\boldeta = \boldzeta = \boldtheta$
and $\boldeta = \boldzeta = \boldvarphi$.
In these cases the computer's part of \eqref{eq:chain-four} is trivial.

\subsection{Conditional Canonical to Unconditional Canonical}

This section deals with $f_{\boldtheta, \boldvarphi}$, which is described
by \eqref{eq:new-iid}.
Let $\Delta \theta_{i j}$ denote an increment in $\theta_{i j}$
and similarly for $\Delta \varphi_{i j}$.  Then
\begin{equation} \label{eq:theta-to-phi}
   \Delta \varphi_{i j}
   =
   \Delta \theta_{i j}
   -
   \sum_{k \in S(j)} \psi_k'(\hat{\theta}_{i k}) \Delta \theta_{i k}
\end{equation}
provides a description of $\nabla f_{\boldtheta, \boldvarphi}(\boldthetahat)$.
It is a linear operator that
maps a vector with components $\Delta \theta_{i j}$
to a vector with components $\Delta \varphi_{i j}$.

It is perhaps easier to understand this (and more useful to the actual
computer programming of predictions) if we compute the composition
$\nabla f_{\boldeta, \boldzeta}(\boldetahat) \circ \boldM$, the last
two bits of \eqref{eq:chain-four}.  We get, in this case where
$\boldeta = \boldtheta$,
$$
   \frac{\partial \varphi_{i j}}{\partial \beta_k}
   =
   m_{i j k}
   -
   \sum_{l \in S(j)} \psi_l'(\hat{\theta}_{i l}) m_{i l k}
$$
where $m_{i j k}$ are the components of $\boldM$.

\subsection{Unconditional Canonical to Conditional Canonical}

This section deals with $f_{\boldvarphi, \boldtheta}$, which is implicitly
described by \eqref{eq:new-iid}.
It is clear that inverting \eqref{eq:theta-to-phi} gives
\begin{equation} \label{eq:phi-to-theta}
   \Delta \theta_{i j}
   =
   \Delta \varphi_{i j}
   +
   \sum_{k \in S(j)} \psi_k'(\hat{\theta}_{i k}) \Delta \theta_{i k}
\end{equation}

Since \eqref{eq:phi-to-theta} has $\Delta \theta_{i m}$ terms on both
sides, it must be used recursively, as with many other aster
model equations, including \eqref{eq:new-iid} itself.
Clearly, if \eqref{eq:phi-to-theta} is used when $S(j)$ is empty,
it is trivial.  This gives us \eqref{eq:phi-to-theta} for all ``leaf''
notes of the graph.  We can then use \eqref{eq:phi-to-theta} for $j$
such that $S(j)$ is contained in the leaf nodes.  This gives us more
nodes done, and we can repeat, at each stage being able to use
\eqref{eq:phi-to-theta} for $j$ such that $S(j)$ is contained in the
set of nodes already done.  When the graphical model is drawn,
like Figure~\ref{fig:graph}, with
parents above children then the recursion moves up the graph
from children to parents to grandparents and so forth.

If $\boldG = \nabla f_{\boldeta, \boldzeta}(\boldetahat) \circ \boldM$
had components $g_{i j k}$ and $\boldM$ has components $m_{i j k}$, then
we get in this case where $\boldeta = \boldvarphi$,
\begin{equation} \label{eq:D-beta-to-phi-to-theta}
   g_{i j k}
   =
   m_{i j k}
   +
   \sum_{l \in S(j)} \psi_l'(\hat{\theta}_{i l}) g_{i l k}
\end{equation}
and (as discussed above), since $g_{i \cdot k}$ is on both sides of
the equation \eqref{eq:D-beta-to-phi-to-theta} must be used recursively
going from the leaves of the graph toward the roots.

\subsection{Conditional Canonical to Conditional Mean Value}

This section deals with $f_{\boldtheta, \boldxi}$, which is trivial
given exponential family theory.  This map is given by
equation \eqref{eq:mvp-cond} which we repeat here
$$
   \xi_{i j} = X_{i p(j)} \psi_j'(\theta_{i j})
$$
and so the derivative is trivially
\begin{equation} \label{eq:theta-to-xi}
   \frac{\partial \xi_{i j}}{\partial \theta_{i j}}
   =
   X_{i p(j)} \psi_j''(\theta_{i j})
\end{equation}
(other partial derivatives being zero).

\subsection{Unconditional Canonical to Unconditional Mean Value}

This section deals with $f_{\boldvarphi, \boldtau}$, which is also trivial
given exponential family theory.  This map is given by
equation \eqref{eq:mvp-exp} which we repeat here
$$
   \boldtau
   =
   f_{\boldvarphi, \boldtau}(\boldvarphi)
   =
   \nabla \psi(\boldvarphi)
$$
and so the derivative is trivially
\begin{equation} \label{eq:phi-to-tau}
   \nabla f_{\boldvarphi, \boldtau}(\boldvarphi)
   =
   \nabla^2 \psi(\boldvarphi).
\end{equation}
Although algebraically complicated, this map is already known to
the computer, since it needs it to calculate Fisher information
for unconditional models.  It is completely described
by equations \eqref{eq:fish-food} and \eqref{eq:fish-hook} in
Section~\ref{sec:FEFth}.

\subsection{Conditional Canonical to Unconditional Mean Value}

This section deals with $f_{\boldtheta, \boldtau}$, which could be
considered already done because
of $f_{\boldtheta, \boldtau}
= f_{\boldvarphi, \boldtau} \circ f_{\boldtheta, \boldvarphi}$.
We could compute derivatives using the chain rule.

But let us try something different.  We have a simple expression of
$\boldtau$ in terms of $\boldtheta$ given by \eqref{eq:mvp-exp} in
Chapter~\ref{ch:aster}
$$
   \tau_{i j}(\boldtheta)
   =
   X_{i f(j)}
   \prod_{\substack{m \in J \\ j \preceq m \prec f(j)}}
   \psi_m'(\theta_{i m}).
$$
We can easily differentiate this directly
$$
   \frac{\partial  \tau_{i j}(\boldtheta)}{\partial \theta_{i m}}
   =
   \tau_{i j}(\boldtheta)
   \frac{\psi_m''(\theta_{i m})}{ \psi_m'(\theta_{i m})},
   \qquad j \preceq m \prec f(j)
$$
(and other partial derivatives are zero).

\subsection{Unconditional Canonical to Conditional Mean Value}

This section deals with $f_{\boldvarphi, \boldxi}$, and this one
we probably should consider already done using $f_{\boldvarphi, \boldxi}
= f_{\boldtheta, \boldxi} \circ f_{\boldvarphi, \boldtheta}$.
We compute derivatives using the chain rule
$$
   \nabla f_{\boldvarphi, \boldxi}(\boldvarphihat)
   =
   \nabla f_{\boldtheta, \boldxi}(\boldthetahat)
   \circ
   \nabla f_{\boldvarphi, \boldtheta}(\boldvarphihat)
$$
since one of these $\nabla f_{\boldtheta, \boldxi}(\boldthetahat)$
is diagonal, given by \eqref{eq:theta-to-xi}, this should be easy.
The other bit $\nabla f_{\boldvarphi, \boldtheta}(\boldvarphihat)$
is given by \eqref{eq:D-beta-to-phi-to-theta} and the following discussion.

\section{Discussion}

All of this is a bit brief, but it is the design document that was used
to implement and test the code in the \verb@aster@ package for R.


\chapter{Some One-Parameter Exponential Families for Aster Models}
\label{ch:unit}


\section{Bernoulli}

\subsection{Density}

$$
   f_p(x) = p^x (1 - p)^{1 - x}
$$

\subsection{Canonical Parameter}

The log density is
\begin{align*}
   \log f_p(x)
   & = 
   x \log p + (1 - x) \log (1 - p)
   \\
   & = 
   x \log \left(\frac{p}{1 - p}\right) + \log (1 - p)
\end{align*}
from which it is seen that the canonical parameter is
$$
   \theta = \logit(p) = \log \left(\frac{p}{1 - p}\right)
$$
Note that the inverse map is
$$
   p = \logit^{-1}(\theta) = \frac{1}{1 + e^{- \theta}}
$$

\subsection{Cumulant Function}

We must have
$$
   \log f_p(x) = x \theta - \psi(\theta) + h(x)
$$
from which we get
\begin{align*}
   \psi(\theta)
   & =
   - \log(1 - p)
   \\
   & =
   - \log \left( \frac{1}{1 + e^\theta} \right)
   \\
   & =
   \log \left( 1 + e^\theta \right)
\end{align*}

\subsection{Mean Function}

By \emph{mean function} we mean the map between the canonical parameter
and the mean value parameter
$$
   \tau(\theta) = E_\theta X = \psi'(\theta).
$$
Here
$$
   \tau(\theta) = \frac{1}{1 + e^{- \theta}}
$$

\subsection{Variance Function}

By \emph{variance function} we mean the derivative of $\tau$
$$
   \nu(\theta) = \tau'(\theta) = \var_\theta X = \psi''(\theta).
$$
Here
$$
   \nu(\theta) = \frac{e^{- \theta}}{(1 + e^{- \theta})^2}
$$

\subsection{Check}

We do have
\begin{align*}
   \tau(\theta) & = p
   \\
   \nu(\theta) & = p (1 - p)
\end{align*}
the familiar mean and variance of the Bernoulli distribution.

\section{Poisson}

\subsection{Density}

$$
   f_\mu(x) = \frac{\mu^x}{x !} e^{- \mu}
$$

\subsection{Canonical Parameter}

The log density is
\begin{align*}
   \log f_\mu(x)
   & =
   x \log \mu - \mu - \log (x !)
\end{align*}
from which it is seen that the canonical parameter is
\begin{subequations}
\begin{equation} \label{eq:fred}
   \theta = \log(\mu)
\end{equation}
Note that the inverse map is
\begin{equation} \label{eq:fred-inv}
   \mu = e^\theta
\end{equation}
\end{subequations}

\subsection{Cumulant Function}

And
\begin{align*}
   \psi(\theta)
   & =
   \mu
   \\
   & =
   e^\theta
\end{align*}

\subsection{Mean Function}

And
$$
   \tau(\theta) = e^\theta
$$

\subsection{Variance Function}

And
$$
   \nu(\theta) = e^\theta
$$

\subsection{Check}

We do have
\begin{align*}
   \tau(\theta) & = \mu
   \\
   \nu(\theta) & = \mu
\end{align*}
the familiar mean and variance of the Poisson distribution.

\section{Poisson Conditioned on Non-Zero}

\subsection{Density}

\begin{align*}
   f_\mu(x)
   & =
   \frac{\mu^x}{x !} \cdot \frac{e^{- \mu}}{1 - e^{- \mu}}
   \\
   & =
   \frac{\mu^x}{x !} \cdot \frac{1}{e^\mu - 1}
\end{align*}
($\mu$ is the mean of the Poisson distribution not this distribution).
%%%
%%% <<Statistics`DiscreteDistributions`
%%% dist = PoissonDistribution[mu]
%%% den[x_, mu_] = PDF[dist, x]
%%% con[mu_] = Sum[den[x, mu], {x, 1, Infinity}]
%%%
%%% newden[x_, mu_] = den[x, mu] / con[mu]

\subsection{Canonical Parameter}

The log density is
\begin{align*}
   \log f_\mu(x)
   & =
   x \log \mu - \log \left( e^\mu - 1 \right) - \log (x !)
\end{align*}
from which it is seen that the canonical parameter and its
inverse are still given by \eqref{eq:fred} and \eqref{eq:fred-inv}.

\subsection{Cumulant Function}

And
\begin{align*}
   \psi(\theta)
   & =
   \log \left( e^\mu - 1 \right)
   \\
   & =
   \log \left( e^{e^\theta} - 1 \right)
\end{align*}
%%%
%%% newpsi[theta_] = x theta - Log[newden[x, Exp[theta]]]
%%% newpsi[theta_] = PowerExpand[newpsi[theta]]

\subsection{Mean Function}

Here
\begin{align*}
   \tau(\theta)
   & =
   \frac{e^{e^\theta} e^\theta}{e^{e^\theta} - 1}
   \\
   & =
   \frac{e^\theta}{1 - e^{- e^\theta}}
\end{align*}
%%%
%%% tau[theta_] = D[newpsi[theta], theta]
%%% Limit[tau[theta], theta -> - Infinity ]
%%% Plot[tau[theta], {theta, -3, 3} ]


\subsection{Variance Function}

And
\begin{align*}
   \nu(\theta)
   & =
   \frac{e^\theta}{1 - e^{- e^\theta}}
   -
   \frac{e^{- e^\theta} e^{2 \theta}}{(1 - e^{- e^\theta})^2}
   \\
   & =
   \tau(\theta) \left(1 - \tau(\theta) e^{- e^\theta} \right)
\end{align*}
%%%
%%% nu[theta_] = D[tau[theta], theta]
%%% nu[Log[mu]]
%%% nu[theta] - tau[theta] (1 - tau[theta] Exp[- Exp[theta]])
%%% Plot[nu[theta], {theta, -3, 3} ]
%%% Plot[nu[theta], {theta, -5, 1} ]

\subsection{Check}

We do have
\begin{align*}
   \tau(\theta) & = \frac{\mu}{1 - e^{- \mu}}
   \\
   \nu(\theta) & = \frac{\mu [1 - (1 + \mu) e^{- \mu}]}{(1 - e^{- \mu})^2}
   \\
   & = \tau(\theta) [ 1 - \tau(\theta) e^{- \mu} ]
\end{align*}
the somewhat unfamiliar mean and variance of the Poisson distribution
conditioned on not being zero.
%%%
%%% newmu = Sum[ x newden[x, mu], {x, 1, Infinity} ]
%%% newmu - tau[Log[mu]]
%%% newvar = Sum[ (x - newmu)^2 newden[x, mu], {x, 1, Infinity} ]
%%% newvar = Sum[ (x - newmu)^2 newden[x, mu], {x, 1, Infinity} ]
%%% newvar = Simplify[%]
%%% newvar - (mu (1 - (1 + mu) Exp[- mu])) / (1 - Exp[- mu])^2
%%%
%%% tau[Log[mu]] - mu / (1 - Exp[- mu])
%%%
%%% nu[Log[mu]] - tau[Log[mu]] (1 - tau[Log[mu]] Exp[- mu])
%%% nu[Log[mu]] - newvar
Despite it not being obvious from the formula, $\nu(\theta)$ is indeed
nonnegative, as a variance must be, and goes to zero as $\theta$ goes
to minus infinity.



\chapter{Simulation of Poisson Conditioned on Being Nonzero} \label{ch:sim}

\input{sim}

\chapter{Example Analysis of Purple Coneflower
    (\emph{Echinacea angustifolia}) Data} \label{ch:coneflr}

\input{coneflr}

\chapter{Canonical Parameter Spaces of Aster Models} \label{ch:spaces}


\section{Basic Exponential Family Theory}

The \emph{Laplace transform} of a positive measure $\lambda$ on $\real^J$
is \citep[Chapter~7]{barndorff} the function $c : \real^J \to (0, \infty]$
defined by
$$
   c(\boldvarphi) = \int e^{\inner{\boldx, \boldvarphi}} \lambda(d \boldx).
$$
(note that the value $+ \infty$ is allowed so the function is defined for
all $\boldvarphi$).  A log Laplace transform is both convex and lower
semicontinuous \citep[Theorem~7.1]{barndorff}.  This implies that
$$
   \Phi = \set{ \boldvarphi \in \real^J : c(\boldvarphi) < \infty }
$$
is a convex set.

The full standard exponential family generated by $\lambda$ is the family
$$
   \mathcal{P} = \set{ P_\boldvarphi :  \boldvarphi \in \Phi }
$$
where $P_\boldvarphi$ is the distribution having density with respect to
$\lambda$ defined by
$$
   f_\boldvarphi(x) = \frac{1}{c(\boldvarphi)} e^{\inner{\boldx, \boldvarphi}},
   \qquad \boldvarphi \in \Phi
$$
\citep[Chapter~8]{barndorff}.  The moment generating function
of $P_\boldvarphi$ is defined by
$$
   M_\boldvarphi(t)
   =
   \int e^{\inner{\boldx, t}} P_\boldvarphi(d \boldx)
   =
   \frac{c(\boldvarphi + t)}{c(\boldvarphi)}
$$
From this we see that a Laplace transform is just like a moment generating
function, except for a general positive measure $\lambda$ instead of for a
probability measure $P_\boldvarphi$.

The \emph{cumulant generating function} is the log of the moment generating
function.  Its derivatives evaluated at $t = 0$ are the \emph{cumulants}
of the distribution (the first two are mean and variance).  Since the
derivatives of $\log M_\varphi$ evaluated at $t = 0$ are the same as the
derivatives of $\psi = \log c$ evaluated at $\boldvarphi$, we call $\psi$
the \emph{cumulant function} of the family.

It is sometimes convenient to choose for the dominating measure of the family
($\lambda$ in our original notation) one of the distributions in the family,
a $P_{\boldvarphi^*}$ for some $\boldvarphi^* \in \Phi$.
We need to see what that does to our
original formulation with $\lambda$ as the dominating measure.

The density of $P_\boldvarphi$ with respect to $P_{\boldvarphi^*}$ is
just the ratio of their densities with respect to $\lambda$
\begin{equation} \label{eq:move}
   g_\boldvarphi(x)
   =
   \frac{f_\boldvarphi(x)}{f_{\boldvarphi^*}(x)}
   =
   \frac{c(\boldvarphi^*)}{c(\boldvarphi)}
   e^{\inner{x, \boldvarphi - \boldvarphi^*}}
\end{equation}
and we see that the Laplace transform for this ``new'' family is
$c(\boldvarphi) / c(\boldvarphi^*)$ and the cumulant function is
$\psi(\boldvarphi) - \psi(\boldvarphi^*)$.

\section{Exponential Family Theory Applied to Aster Models}

\subsection{Cumulant Function and Full Canonical Parameter Space}

This appendix investigates what happens when the canonical parameter spaces
of an aster model are not all of $\real^d$.  The full \emph{conditional}
canonical parameter space is a Cartesian product
$$
   \Theta = \prod_{j \in J} \Theta_j
$$
where $\Theta_j$ is the full canonical parameter space of the
conditional one-parameter exponential family model at the $j$-th node
(which has cumulant function $\psi_j$), the set
$$
   \Theta_j = \set{ \theta \in \real : \psi_j(\theta) < \infty }.
$$

Let $P_{\theta, j}$ be the conditional probability measure of the $j$-th
family.  So the joint distribution is
$$
   P_{\boldtheta}(d \boldx)
   =
   \prod_{j \in J} P_{\theta_j, j}(d x_j \mid x_{p(j)}).
$$

Fix $\boldtheta^* \in \Theta$ and let $\boldvarphi^*$ be the corresponding
unconditional canonical parameter found by applying the map \eqref{eq:new}
to $\boldtheta^*$.  Write $\boldvarphi = \boldvarphi^* + \bolddelta$ for
a general unconditional canonical parameter.  The Laplace transform
(which in this case is also a moment generating function)
of $P_{\boldvarphi^*}$ is
\begin{equation} \label{eq:this-is-it}
   \int e^{\inner{\boldx, \bolddelta}} P_{\boldvarphi^*}(d \boldx)
   =
   \exp\bigl( \psi(\boldvarphi^* + \bolddelta) - \psi(\boldvarphi^*) \bigr),
\end{equation}
which agrees with the analysis in \eqref{eq:move}.  Now we define the
full canonical parameter space
$$
   \Phi = \set{ \boldvarphi \in \real^J : \psi(\boldvarphi) < \infty }
$$
(we are using the original parameter $\boldvarphi$ instead of the ``new''
parameter $\bolddelta$).  It stands to reason that $\Phi$ is just the
set of points obtained by mapping $\Theta$ through the change of parameter
defined by \eqref{eq:new}.  The whole point of this appendix is to show
that what seems obvious actually is obvious.

\subsection{Leaf Nodes}

For $j$ a leaf node, the only part of the integral in \eqref{eq:this-is-it}
involving $x_j$ is
$$
   \int e^{x_j \delta_j} P_{\theta_j^*, j}(d x_j \mid x_{p(j)}).
$$
Now from the uniparameter case of the exponential family theory embodied
in \eqref{eq:this-is-it} we see that
$$
   \int e^{x_j \delta_j} P_{\theta_j^*, j}(d x_j \mid x_{p(j)} = 1)
   =
   \exp\bigl( \psi_j(\theta_j^* + \delta_j) - \psi_j(\theta_j^*) \bigr),
$$
from which it follows from the multiplication rule for moment generating
functions and the structure of aster models that
\begin{equation} \label{eq:leaf}
   \int e^{x_j \delta_j} P_{\theta_j^*, j}(d x_j \mid x_{p(j)})
   =
   \exp\bigl( x_{p(j)}
   [ \psi_j(\theta_j^* + \delta_j) - \psi_j(\theta_j^*) ] \bigr).
\end{equation}
This is finite if and only if
$$
   \varphi_j = \theta_j = \theta_j^* + \delta_j
   = \varphi_j^* + \delta_j \in \Theta_j
$$
in short if
$$
   \varphi_j = \theta_j \in \Theta_j.
$$
So that is the finiteness condition for leaf nodes.  The part of the integral
\eqref{eq:this-is-it} that pertains to $\theta_j$ and $\varphi_j$
has the ``obvious'' condition for being finite.

\subsection{Non-Leaf Nodes}

Now consider a non-leaf node $j$ whose children are all leaf nodes.
After integrating out the $x_m$ for all child nodes $m$, as above,
the only part of the integral in \eqref{eq:this-is-it} that contains
$x_j$ is
\begin{subequations}
\begin{equation} \label{eq:phred}
   \int \exp\left( x_j \left[ \delta_j + \sum_{m \in S(j)}
     \bigl( \psi_m(\theta_m) - \psi_m(\theta_m^*) \bigr) \right] \right)
   P_{\theta_j^*, j}(d x_j \mid x_{p(j)})
\end{equation}
where we have written $\theta_m = \varphi_m = \theta_m^* + \delta_m$
for the $m$, all of which are leaf nodes.
We also write $\varphi_j = \varphi_j^* + \delta_j$ and note that from
\eqref{eq:new} we have
$$
   \varphi_j^* + \delta_j + \sum_{m \in S(j)} \psi_m(\theta_m)
   =
   \varphi_j + \sum_{m \in S(j)} \psi_m(\theta_m)
   = \theta_j
$$
and similarly
$$
   \varphi_j^* + \sum_{m \in S(j)} \psi_m(\theta_m^*)
   =
   \theta_j^*
$$
so the term in large square brackets in \eqref{eq:phred}
is just $\theta_j^* - \theta_j$.  Hence \eqref{eq:phred} is
\begin{equation} \label{eq:phred-junior}
   \int \exp\left( x_j \left[ \theta_j^* - \theta_j \right] \right)
   P_{\theta_j^*, j}(d x_j \mid x_{p(j)})
   =
   \exp\bigl( x_{p(j)}
   [ \psi_j(\theta_j) - \psi_j(\theta_j^*) ] \bigr).
\end{equation}
\end{subequations}
and we see that \eqref{eq:phred} and \eqref{eq:phred-junior} say exactly
the same thing as \eqref{eq:leaf}.

However, the implication about finiteness of bits of \eqref{eq:this-is-it}
is a bit different than the case for leaf nodes.  Now we have that
$$
   \theta_j \in \Theta_j
$$
is the finiteness condition for the term on the right hand side of
\eqref{eq:phred-junior} (no surprise there) and that translates to
the following about $\varphi_j$.  Since
\begin{align*}
   \theta_j
   & =
   \varphi_j + \sum_{m \in S(j)} \psi_m(\theta_m)
\end{align*}
the finiteness condition for this section is
\begin{equation} \label{eq:finite}
   \varphi_j \in \Theta_j - \sum_{m \in S(j)} \psi_m(\theta_m).
\end{equation}

\subsection{All Nodes by Mathematical Induction}

Now assume (the induction hypothesis) that
\begin{multline}
   \label{eq:induce}
   \int \exp\left( x_j \left[ \delta_j + \sum_{m \in S(j)}
     \bigl( \psi_m(\theta_m) - \psi_m(\theta_m^*) \bigr) \right] \right)
   P_{\theta_j^*, j}(d x_j \mid x_{p(j)})
   \\
   =
   \exp\bigl( x_{p(j)}
   [ \psi_j(\theta_j) - \psi_j(\theta_j^*) ] \bigr).
\end{multline}
As we have seen, this holds for all of the nodes we have examined so far.
But then we see that by the argument in the preceding section that if
\eqref{eq:induce} holds for all children (successors) of a node, then it
holds for the node itself.

Thus \eqref{eq:finite} is the finiteness condition for all nodes,
including leaf nodes for which $S(j)$ is the empty set.

Hence $\Phi$ is indeed the image of $\Theta$ under the map \eqref{eq:new}.
And the argument did turn out to be ``obvious''.

\subsection{Convexity}

The only non-obvious fact in this whole appendix is that hence $\Phi$ must
be a convex set (like all full canonical parameter spaces).  Moreover,
since the map \eqref{eq:new} is a diffeomorphism (both it and its inverse
are differentiable) between the interiors of $\Theta$ and $\Phi$, it follows
that $\Theta$ and $\Phi$ are both open sets or both not open.

\subsection{Regularity}

The important special case where the full canonical parameter space is an
open set is called \emph{regular} (the family is a \emph{regular} exponential
family) \citep[p.~116]{barndorff}.  They are the most well-behaved with respect
to maximum likelihood (no boundaries of the parameter space to worry about).

Since $\Theta$ is open if and only if each $\Theta_j$ is, we see that
the full flat exponential family (with the unconditional canonical
parameterization) is regular if and only if each one-parameter conditional
family is regular.

\subsection{Steepness}

A full exponential family is \emph{steep} \cite[p.~117]{barndorff} if
given $\boldvarphi_i$ in the interior of $\Phi$ and $\boldvarphi_b$ on the
boundary of $\Phi$, then
$$
   \inner{\nabla \psi\bigl(t \boldvarphi_i + (1 - t) \boldvarphi_b \bigr),
   \boldvarphi_b - \boldvarphi_i} \to \infty,
   \qquad \text{as $t \downarrow 0$}.
$$
Clearly this carries over to any flat subfamily (formed by intersecting
the full canonical parameter space $\Phi$ with an affine subspace).
The subfamily is steep if the full family is.  Thus we concentrate on
the full (FEF) family only.

An equivalent (if and only if) condition for steepness is that the MLE map
$\boldx \mapsto \boldvarphihat(\boldx)$ is one-to-one and is found by
solving the ``likelihood equations'' which have ``observed equals expected''
form, as in equation \eqref{eq:score-uncon} and the preceding unnumbered
equation \citep[Theorem~9.14, Corollary~9.6 and their surrounding
discussion]{barndorff}.

Yet another equivalent (if and only if) condition for steepness is that
the mean value parameterization
map, $\nabla \psi : \boldvarphi \mapsto \boldtau$ in aster model notation,
maps the interior of the full canonical parameter space $\Phi$ onto the
interior of the closed convex support of the canonical statistic
\citep[pp.~117, 142, and 152]{barndorff}.

Let $a_j$ and $b_j$ be the endpoints (possibly infinite) of the full
mean-value parameter space of the one-parameter exponential family
for the $j$-th node.  The endpoints themselves may or may not be in
the mean-value parameter space
$$
   \set{ \psi_j'(\theta) : \theta \in \Theta_j }
$$
but the closed convex support of the $j$-th one-parameter exponential
family is the closed interval
$\real \cap [a_j, b_j]$ if this $j$-th family is steep
(the point of the $\real \cap \vphantom{a_j}$ is
to omit $- \infty$ and $+ \infty$
if either $a_j = - \infty$ or $b_j = + \infty$).

Clearly from equation \eqref{eq:mvp-exp} the closed convex support of
$X_j$ in the FEF is the closed interval
\begin{equation} \label{eq:sally}
   \real \cap
   \left[
   X_{i f(j)}
   \prod_{\substack{m \in J \\ j \preceq m \prec f(j)}} a_m,
   X_{i f(j)}
   \prod_{\substack{m \in J \\ j \preceq m \prec f(j)}} b_m
   \right].
\end{equation}
It is clear also that as $\boldtheta$ runs over
the interior of $\Theta$ the
mean value parameter $\boldtau(\boldtheta)$ runs over
the interior of \eqref{eq:sally}.  Hence if each of the
one-parameter exponential families is steep, then the FEF of an
aster model is steep.

\section{Conclusions}

This whole appendix is (in hindsight) ``obvious.''  What we have learned
is that what we thought was obvious is indeed obvious and there is no
problem with maximum likelihood in an aster model if each of the one-parameter
exponential families is regular or steep.  Moreover, we have learned that
the canonical parameter space of the FEF is convex (like every full canonical
parameter space of every full exponential family) something that is ``obvious''
only from exponential family theory and not from looking at the definition
of the map \eqref{eq:new} between conditional and unconditional canonical
parameterizations.



\end{document}

